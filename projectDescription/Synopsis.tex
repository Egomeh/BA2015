\documentclass[11pt,a4paper]{article}
\usepackage{a4wide}
\usepackage{natbib}
\usepackage{graphicx}
\usepackage{wrapfig}
\usepackage{amsfonts}

\usepackage{hyperref}
\usepackage{xcolor}
\usepackage[tikz]{bclogo}
\usepackage[framemethod=tikz]{mdframed}

\mdfdefinestyle{mystyle}{%
  rightline=true,
  innerleftmargin=10,
  innerrightmargin=10,
  outerlinewidth=3pt,
  topline=false,
  rightline=true,
  bottomline=false,
  skipabove=\topsep,
  skipbelow=\topsep
}

\setlength{\parindent}{0cm}

\begin{document}

\title{BSc Project Description\\\textbf{Learning to Play Tetris using
    the Covariance Matrix Adaptation
    Evolution Strategy}}
\date{\today}
\maketitle

% Removed because Motivation and background is the same
%\section{Background}


\subsection{Previous work in learning Tetris\label{prevWork}}
Over the time, numerous researchers has tried different feature 
sets and applied various optimizers in the search of the best 
possible policy. Researches in the field
of reinforcement learning have approached the task of learning Tetris
in various ways. Some have implemented exact versions of the original game
where the artificial Tetris player will have to interact with the 
game much like a human player. In these games the state of the board is not limited
to just the board configuration and the current piece, but also where in the 
board the piece is located. Hence, these controllers does not model actions 
as locations to drop the piece, but rather 'button presses' that controls the
fall of the piece. Yet, from the literature we have read, the most commonly applied 
type of controller is the one used in the MDPTetris platform. As mentioned earlier,
to create an efficient controller one can adjust two settings, namely the feature set 
and the associated weights. 
Usually, other reseachers deals with the combined task of both
deciding the feature set and tuning the weights. A common 
feature set, the Dellacherie feature set (see figure \ref{table:dellfeat}),
was originally tuned by hand with trial and error approach with surprisingly 
good results. Yet, the most common approach is to apply an optimization 
algorithm to adjust the weighting.
The features used are typically
ones that attempt to mimic the board conditions that would
normally catch the attention of a human player, such as
how high the overall pile of pieces is and how many holes 
the board has. Table 1 \citep{scherrer2009:b}
presents some feature sets used throughout various publications
on the subject. The feature sets applied in this thesis are 
Dellacherie and the Bertsekas feature sets 
(see figure \ref{table:dellfeat} and \ref{table:bertfeat}) due 
to their recurring usage across various articles.
Many authors have had success
with applying evolutionary stochastic search methods for tuning 
the weights of the feature sets towards
efficient controllers. However, the goal of most research 
is to outperform existing controllers and push the boundaries
of performance when learning Tetris. This means that the objective of
these researchers is often to craft feature sets that perceives the 
board in a way that allows the agent to make the best possible decisions.
Thus the focus in most works is combined on finding good feature sets and 
finding good ways to optimize the weights of the set. In this thesis,
our goal is not to find a controller that outperforms any existing one,
but rather to investigate the learning properties of the optimization algorithms.
For this purpose
we are in particular addressing the
Cross-entropy method described in detail in \citep{cetut2014} and the
Covariance Matrix Adaption Evolution Strategy (CMA-ES), described 
in \citep{hansen2011}. The particular Cross-entropy method applied 
is the one described in \citep{szita:06} as the "Noisy Cross-entropy Method".\\

\begin{table}[h!]
\begin{center}
\begin{tabular}{| l | p{8cm} |}
\hline
\textbf{Feature} & \textbf{Description}\\
\hline
Landing height & The height of the last piece when it was placed\\
\hline
Eroded piece cells & Number of rows cleared in the last move
times the number of bricks cleared from the last move\\
\hline
Row transitions & Number of horizontal cell transitions\\
\hline
Column transitions & Number of vertical cell transitions \\
\hline
Holes & Number of empty cells surrounded by full cells\\
\hline
Board wells & Cumulative sum of cells to the depth of
the board wells.\\
\hline
\end{tabular}
\end{center}
\caption{features of the Dellacherie featureset \label{table:dellfeat}}
\end{table}

\begin{table}[h!]
\begin{center}
\begin{tabular}{| l | p{8cm} |}
\hline
\textbf{Feature} & \textbf{Description}\\
\hline
Max height & Height height of the highest occupied block\\
\hline
Holes & Number of empty cells surrounded by full cells\\
\hline
Column height & Height of each column\\
\hline
Height difference & The height difference between columns\\
\hline
\end{tabular}
\end{center}
\caption{features of the Dellacherie featureset \label{table:bertfeat}}
\end{table}


Currently, many researchers have proposed numerous 
feature sets and multiple 
optimization methods have been explored. 
The Dellacherie controller is very widely used across many resources
\citep{fahey}. This controller was hand-tuned by trial and error 
and did originally achieve, on a regular non-simplified Tetris game, an average of
660,000 lines. The same feature set (see figure \ref{table:dellfeat}) is 
often incorporated in later works when optimizing controllers. An earlier
feature set is the set proposed by \citep{Bertsekas} referred to as the Bertsekas and
Tsitsiklis features. In 2006, Szita and L\H{o}rincz \citep{szita:06} applied the Cross-entropy
method using the Bertsekas and Tsitsiklis features. They report that using no noise,
their controller converged at 300,000 lines on average. 
The best result reported in \citep{szita:06}
is when decreasing noise is applied, 
in which the controller's score exceeded 800,000 lines. 
However, in a later paper, using Dellacherie, 
Bertsekas and two selfdefined features achieved 
35,000,000 lines $\pm 20\%$  \citep{scherrer2009}.\\




\input{Background/objectiveFunction}

\subsection{Optimizers \label{Optimizers}}

As mentioned earlier, both the Cross-entropy method and CMA-ES fall into the category of 
\textit{stochastic optimization}
methods. These methods are useful for 
optimization problems where the gradient is not available.
The optimization functions aim to optimize 
the parameter set $\textbf{\individual }$
for the objective function $\fitnessFunction$.
\begin{align}
\hat{\textbf{\individual }} &= 
arg \  \underset{\textbf{\individual }}{max} \  
\fitnessFunction (\textbf{\individual }) \ 
:\mathbb{R}^{\dimensions } \rightarrow \mathbb{R}
\end{align}

Often, the problem is depicted as a minimization problem which is quite
the opposite in relation to Tetris where the goal is to gain as high score as possible.
Therefore we depict the problem entirely as a optimization problem.
In these optimization methods, the optimizing algorithm uses a family of parametric distributions,
and maintains a set of parameters for the used distribution
to search the best possible solution for the objective function.  
In the case studied in this thesis
both, the CMA-ES and Cross-entropy method use a 
Gaussian distribution to sample solutions to the objective function.
Hence, both functions aim to find a mean 
$\mean $ and an $\dimensions \times \dimensions$ matrix 
$\varianceMatrix $\footnote{In \citep{hansen2011}, 
$\sigma$ is used for step-size in CMA-ES, so $\varianceMatrix $ is instead introduced
as an arbitrary $\dimensions \times \dimensions$ matrix in its place.}, such that when
a vector $\textbf{\individual }$ is sampled by 
$\textbf{\individual } \sim \mathcal{N}\left( \mean, \varianceMatrix \right)$, 
then $\fitnessFunction (\textbf{\individual })$ 
is likely to yield the desired results.\\
\\
The algorithms work iteratively, such that the mean and variance 
of the distribution 
is altered for each iteration $\generation $.
The algorithms start by initializing the 
parameters either at random or at some fixed point. A common 
configuration is setting the mean to 
a zero-vector and the standard deviation to the identity matrix.
Thus, for the first iteration $\generation = 0$, a configuration could be as follows
\begin{align}
\mean^{(0)} =
\begin{bmatrix}
0\\
\vdots\\
0
\end{bmatrix},\ \ 
\varianceMatrix^{(0)} = 
\begin{bmatrix}
1 & \hdots & 0\\
\vdots & \ddots & \vdots\\
0 & \hdots & 1
\end{bmatrix}
\end{align}

The superscript of $(0)$ notes that the values occur in generation 0.\\
\\
In each generation, the algorithms sample $\populationSize$ search points
and evaluate their fitness
against the objective function. When each of the solutions are evaluated,
they are ordered according to their fitness
\begin{align}
\{\textbf{\individual }_{1}, \hdots, 
\textbf{\individual }_{\populationSize }\}\ \ \text{Such that}\ \ 
\fitnessFunction(\textbf{x}_1) \geq 
\fitnessFunction(\textbf{\individual }_2), \hdots, 
\fitnessFunction(\textbf{\individual }_{\populationSize  - 1}) \geq 
\fitnessFunction(\textbf{\individual }_{\populationSize })
\end{align}
This sorting is a rather essential part of how the search parameters are
updated. Both the Cross-entropy method and CMA-ES algorithm use a fraction of the 
best solutions, that were drawn from its Gaussian distribution. An essential 
part of the selection is that the candidate solutions are not contributing
directly according to their yield from the \textit{fitness function}, but rather
how they are ranked relative to the other solutions.
The mean and standard deviation for the next iteration,
$m^{(\generation+1)}$ and $M^{(\generation+1)}$, are
then updated usually by considering the best of the ordered solutions. How exactly
these parameters are updated is individual for each method is elaborated further in the
following sections.




\section{CE (Cross Entropy) \label{CrossEntropy}}
CE is described through many papers in 
slightly different ways. Using similar 
format as in the paper \citep{thiery:09}.\\
\\
This method uses a Gaussian distribution and 
attempts to find distribution parameters 
that yields good candidates for the 
objective function $\fitnessFunction$.\\
\\
The Cross-Entropy method starts with an initial 
mean $\mean$ and standard deviation $\varianceMatrix$. 
The mean is usually an $\dimensions$ dimensional vector
set to:
\begin{align*}
\mean = \begin{pmatrix}
0\\
\vdots\\
0
\end{pmatrix} 
\end{align*}

The variance is kept individual for each dimension, 
and is usually initialized as follows:

\begin{align*}
\varianceMatrix =
\begin{bmatrix}
\sigma_1 & \hdots & 0\\
\vdots & \ddots & \vdots\\
0 & \hdots & \sigma_{\dimensions}
\end{bmatrix}
\end{align*}

Where in this context, $\sigma$ represents \textit{standard deviation}.\\
\\
The algorithm then works iteratively on generations of individual
search points acting as candidate inputs for the objective function.
In each generation, $\populationSize$ vectors are sampled by 
$\individual_{i} \sim \mathcal{N} \left(\mean, \varianceMatrix^{2} \right)
,\ i \in \{1,\dots,\populationSize \}$. The vectors are all evaluated 
against the fitness function and ordered such that $\fitnessFunction \left( \individual_{1} \right) \geq, \dots, \geq \fitnessFunction \left( \individual_{\populationSize} \right)$
, and the $\offspringNumber$ best are chosen for updating the distribution 
parameters. The mean is updated as the centroid of the chosen vectors, and
the variance is updated as the variance of the chosen vector in each 
dimension.\\

The pseudo code and details of the algorithm can be seen in figure
\ref{fig:ceCode} on page \pageref{fig:ceCode}.

\begin{figure}[H]
\hrule
\vspace{0.2cm}
{\centering  \textit{Noisy cross-entropy method}}
\vspace{0.2cm}
\hrule
\begin{algorithmic}
\State{\textbf{input}}
\State{$\fitnessFunction$ : The function that estimates the performance of a vector $\individual$}
\State{($\mean_0$, $\varianceMatrix^2_0$): The mean and variance of the initial distribution}
\State{$\populationSize$ : The number of vectors sampled per generation/iteration}
\State{$\offspringNumber$: The number of offspring selected for the new mean}
\State{$\noise_{\generation}$: The noise added to each generation/iteration}
\\

\Loop
\State{Generate $\populationSize$ vectors $\individual_{1}, \individual_{2}, \dots, \individual_{\populationSize}$ from $\mathcal{N}(\mean_{\generation}, \varianceMatrix^2_{\generation})$}
\State{Evaluate each vector using $\fitnessFunction$}
\State{Select the $\offspringNumber$ vectors with the highest evaluation}
\State{Update $\mean_{\generation + 1}$ of the $\offspringNumber$ best vectors}
\State{Update $\varianceMatrix^2_{\generation + 1}$ of the $\offspringNumber$ best vectors + $\noise_{\generation}$}
\EndLoop
\end{algorithmic}
\hrule
\caption{The pseudo code for the Cross-Entropy algorithm \label{fig:ceCode}}
\end{figure}

\subsection{Input}

\textbf{The objective function \label{CEObjective}} \\
The function used to assess the value of a sampled vector.
As described in the 'Optimizers' section, CE is a general stochastic 
iterative algorithm that tries to solve an optimization problem of 
the form \citep{thiery:09}:
\begin{align*}
\hat{\textbf{\individual }} &= 
arg \  \underset{\textbf{\individual }}{max} \  
\fitnessFunction (\textbf{\individual }) \ 
\end{align*}
Where $x$ corresponds to a given vector, 
and $\fitnessFunction$ is our actual objective function. 
\\

\textbf{The mean and variance of the gaussian distribution} \\
Here $\mean_{\generation}$ is the mean and  
$\varianceMatrix^2_{\generation}$ is the variance 
of the gaussian distribution ($\mean_{\generation}$,
$\varianceMatrix^2_{\generation}$). 
More specifically this gaussian distribution is defined as 
\begin{align*}
\mathcal{N}(\mean_{\generation},\varianceMatrix^2_{\generation})
\end{align*}

Where $\generation$ denotes the current iteration.\\


\textbf{The number of vectors}\\
$\populationSize$ is the number of vectors sampled in each generation.
\\

\textbf{The number of offspring}\\
$\offspringNumber$ is the number of vectors which are used to compute 
the new mean, $\mean_{\generation + 1}$, and variance,
$\varianceMatrix^2_{\generation + 1}$, for next generation/iteration. 
These offspring vectors gets selected 
directly by taking $\offspringNumber$ vectors
which got the best evaluation.
\\

\textbf{The noise factor}\\
The noise factor, $\noise_{\generation}$, is the amount of noise which 
is applied to the variance $\varianceMatrix^2$ in iteration/generation 
$\generation$. In general, noise is used to avoid the risk of a local optimum.
There are different kinds of noise settings, such as: no noise, constant noise 
and linear decreasing noise \citep{szita:06}. 
When using no noise, $\noise_{\generation}$ 
is simply set to zero. When using constant noise, the same value is 
added to the variance $\varianceMatrix^2$ in each iteration/generation. 
When using linear decreasing noise, $\noise_{\generation}$ is defined as
$\noise_{\generation}=max(5- \generation /10,0)$.
\\

\subsection{Loop}

\textbf{Sampling the population}\\
The first step of the loop is to create the new generation consisting of $\populationSize$ vectors. These vectors are sampled randomly within the distribution $\individual_{i}\sim \mathcal{N}(\mean_{\generation},\varianceMatrix^2_{\generation})$.
\\

\textbf{Evaluating the population}\\
After sampling the population, the algorithm needs to order the vectors to find the $\offspringNumber$ best vectors, each vector $\individual_{i}\ ,\ i \in \{1, \dots, \populationSize\}$ is evaluated using $\fitnessFunction$. 
The value from the objective function then yields 
the estimated performance of each individual.
\\

\textbf{Selecting the offspring}\\
As each $\individual_{i}$ has an assigned evaluation value, and the $\offspringNumber$ best vectors gets selected by taking the $\individual_{i}$ vectors with the highest evaluation value.
\\

\textbf{Updating the distribution parameters}\\
When updating the distribution parameters for the next iteration
($\mean_{\generation + 1}$,$\varianceMatrix^2_{\generation + 1}$), 
the mean is updated by computing the centroid of the 
$\offspringNumber$ best vectors. This is formally defined as:
\begin{align*}
\mean_{\generation +1}:=\frac{\sum_{i}^{\offspringNumber} \individual_{i}}{\offspringNumber}
\end{align*}
The variance $\varianceMatrix^2_{\generation + 1}$ is updated 
to match the variance of the $\offspringNumber$ best
vectors, such that the variance in dimension $i$ 
matches the variance of the $\offspringNumber$
in dimension $i$.
This is formally defined as:
\begin{align*}
\varianceMatrix^2_{\generation +1}:=\frac{\sum_{i}^{\offspringNumber}
(\individual_{i} - \mean_{\generation +1})^T(\individual_{i} - \mean_{\generation +1})}{\offspringNumber} + \noise_{\generation + 1}
\end{align*}
\\


\section{CMA-ES}


This description is based on the tutorial written by Nikolaus Hansen
\citep{hansen2011}. The section will not in detail cover how 
the CMA-ES is derived, but rather how it deviates from Cross Entropy.\\
\\
The CMA-ES operates on a general level much like the Cross Entropy 
method, but includes some features that increases the adaptability 
of the algorithm.\\
\\
The CMA-ES, like the Cross Entropy, uses a Gaussian distribution to
search for good solutions to $\fitnessFunction$. Yet, in the second argument 
in the Gaussian distribution, the CMA-ES provides a covariance matrix.
As the Cross Entropy method only provides a diagonal matrix of scalers
it's restricted to only scaling the ellipsoid of equal density along
the coordinate axes. The CMA-ES however, with a full covariance matrix,
allows the ellipsoid to rotate arbitrarily in the search space.\\
\\
While the Cross Entropy only considers the next population when updating the
distribution parameters, the CMA-ES keeps some information from earlier 
generations. This allows the CMA-ES somewhat keep track of the evolution 
of the sampled vectors.\\
\\
The CMA-ES also differs from Cross Entropy in how it evaluates the influence 
of the offspring vectors. As Cross Entropy weights all vectors equally when 
moving the mean. The CMA-ES, at least from the implementation in SHARK, 
has the option of taking a weighted combination of the offspring in order
to bias towards the better vectors.

\begin{figure}[H]
\hrule
\vspace{0.2cm}
{\centering  \textit{CMA-ES}}
\vspace{0.2cm}
\hrule
\begin{algorithmic}
\State{\textbf{input}}
\State{$\fitnessFunction$ : The function that estimates the performance of a vector $\individual$}
\State{($\mean$, $C$): The mean and variance of the initial distribution, where 
$C$ is the covariance matrix usually set to $C=I$}
\State{$\populationSize$ : The number of vectors sampled per generation/iteration}
\State{$\offspringNumber$: The number of offspring selected for the new mean}
\\
\State{\textbf{initialization}}
\State{Set initial internal parameters}
\\
\Loop
\State{Sample new generation}
\State{Evaluate each vector using $\fitnessFunction$ and recombine}
\State{Step-size control}
\State{Covariance matrix adaption}
\EndLoop
\end{algorithmic}
\hrule
\caption{The pseudo code for the Cross-Entropy algorithm \label{fig:ceCode}}
\end{figure}

\comment{All CMA specific beow this is NOT DONE!}

\subsection{Input}

\textbf{The objective function} \\
This serves the same purpose as in Cross Entropy (see page \pageref{CEObjective}).
\\

\textbf{The mean and variance of the gaussian distribution (Adapt to CMA)} \\
Here $\mean_{\generation}$ is the mean and  
$C^{\generation}$ is the variance 
of the gaussian distribution ($\mean_{\generation}$,
$\varianceMatrix^2_{\generation}$). 
More specifically this gaussian distribution is defined as 
\begin{align*}
\mathcal{N}(\mean_{\generation},\varianceMatrix^2_{\generation})
\end{align*}

Where $\generation$ denotes the current iteration.\\


\textbf{The number of vectors}\\
$\populationSize$ is the number of vectors sampled in each generation.
\\

\textbf{The number of offspring}\\
$\offspringNumber$ is the number of vectors which are used to compute 
the new mean, $\mean_{\generation + 1}$, and variance,
$\varianceMatrix^2_{\generation + 1}$, for next generation/iteration. 
These offspring vectors gets selected 
directly by taking $\offspringNumber$ vectors
which got the best evaluation.
\\


\subsection{Initialization}


Set parameters
\begin{align*}
\populationSize, \offspringNumber, w_{i \dots \offspringNumber}, c_{\sigma}, d_{\sigma}, c_c, c_1, c_{\offspringNumber}
\end{align*}
To their default values according to table 1 in \citep{hansen2011}.

Set evolution path $p_{\sigma} = 0$, $p_{c} = 0$, covariance matrix $C = I$ and $\generation = 0$

\subsection{Loop}

\textbf{Sample new generation}\\
Sample new population of search points, for $k = 1, \dots, \populationSize$

\begin{align*}
z_{k} &\sim \mathcal{N}(0, I)\\
y_{k} &= BDz_{k} \sim \mathcal{N}(o, C)\\
x_{k} &= m + \sigma y_{k} \sim \mathcal{N}(m, \sigma^2 C)
\end{align*}

\textbf{Evaluate each vector using $\fitnessFunction$ and recombine}\\
Selection and recombination

\begin{align*}
\langle y \rangle_{w} &= \sum^{\offspringNumber}_{i = 1}w_{i} y_{i : \populationSize}, \text{where} \sum^{\offspringNumber}_{i = 1} w_{i} = 1, w_{i} > 0\\
m &\leftarrow m + \sigma \langle y \rangle_{w} = \sum^{\offspringNumber}_{i = 1} w_{i}x_{i:\populationSize}
\end{align*}

\textbf{Step-size control}

\begin{align*}
p_{\sigma} &\leftarrow \left( 1 - c_{\sigma} \right) p_{\sigma} + \sqrt{c_{\sigma} \left( 2 - c_{\sigma} \right) \offspringNumber_{\text{eff}}} C^{- \frac{1}{2}} \langle y \rangle_{w}\\
\sigma &\leftarrow \sigma \times exp \left( \frac{c_{\sigma}}{d_{\sigma}} \left( \frac{||p_{\sigma}||}{E|| \mathcal{N}(o, I) ||} - 1 \right) \right)
\end{align*}

\textbf{Covariance matrix adaption}

\begin{align*}
p_{c} &\leftarrow \left( 1 - c_{c} \right) p_{c} + h_{\sigma} \sqrt{c_{\sigma} \left( 2 - c_{c} \right) \offspringNumber_{\text{eff}}} \langle y \rangle_{w}\\
C &\leftarrow \left( 1 - c_{c} - c_{\offspringNumber} \right) C + c_{1} \left( p_{c} p_{c}^{T} + \delta \left( h_{\sigma} \right) C \right) + c_{\offspringNumber} \sum^{\offspringNumber}_{i = 1} w_{i} y_{i : \offspringNumber} y_{i : \offspringNumber} ^{T}
\end{align*}









\subsection{Normalization of samples \label{normalSamples}}
As mentioned by some authors \citep{boumaza2009}, the vector that
describes the agent can very well be normalized such that the vector
is a point that lies on the $\dimensions$-dimensional hypersphere.\\
\\
The reason for this lies in the nature of the evaluation function.
When the controller chooses an action, it will evaluate all the 
possible actions possible with the current piece. It will use the 
value function $\valueFunction$ of each state $\gameState_i$ and 
choose the state with the highest value from the value function.
Thus, if the states are ordered such that:
\begin{align*}
\valueFunction \left(  \gameState_1 \right) 
> \dots 
> \valueFunction \left( \gameState_\populationSize \right)
\end{align*}

The agent then chooses the action that transitions from the current state 
to state $\gameState_1$.\\
Since the value function assess the state by the following:
\begin{align*}
\valueFunction (\gameState ) &= 
\sum_{i=1}^{\dimensions} \weight _{i}\feature _{i}(\gameState )
\end{align*}

Then scaling the input of the agent, the weight vector $\weight$, by a
number $a \in \mathbb{R}, \ a > 0$ the assessment is changed by:
\begin{align*}
\sum_{i=1}^{\dimensions} a \weight _{i}\feature _{i}(\gameState ) &= 
a\sum_{i=1}^{\dimensions} \weight _{i}\feature _{i}(\gameState )\\
&= a \valueFunction \left( \gameState \right)
\end{align*}

And the ordering remains:
\begin{align*}
a \valueFunction \left(  \gameState_1 \right) 
> \dots 
> a \valueFunction \left( \gameState_\populationSize \right)
\end{align*}

Thus the order of the value functions of 
each state does not change, and the same $\gameState_1$
is still chosen for any $a \in \mathbb{R}, \ a > 0$.\\
\\
To verify this, the Tetris objective function was executed with the
same vector and the same seed for the random generator with a scale
$a \in \{0.1, 0.2,0.3, \dots, 9.8,9.9,10.0\}$, and the agent scored 
exactly the same for each scale.\\
\\
This can be used in experiments for various reasons. As reported 
in \citep{boumaza2009}, normalizing the samples will 
prevent CMA-ES from diverging in step size,
and it can prevent loosing precision if the magnitude of weight 
vector becomes larger than feasible for the used floating 
point number and avoids size limitations.

\subsection{Assesment of controller performance \label{sec:confidenceIntervals}}
The performance of an one-piece controller has a very high variation,
and is in other papers verified to be exponentially distributed. Hence, the mean
performance of a controller $m$ is exponentially distributed with an unknown 
rate parameter  $\lambda_{exp}$\footnote{Note that $\lambda_{exp}$ refers to the rate parameter
of an exponential distribution, and not the population size of the optimization algorithms.}.
As a result of the high variance of the controllers, the performance 
of single controllers are often presented along with a confidence interval
for the estimated mean score of the controller. The estimated mean of 
the controllers score is computed as
\begin{align}
\hat{\mean} &= \sum_{i=1}^{N} \fitnessFunction \left( \individual \right)
\end{align}
Where $\hat{m}$ is the oberved mean score of the controller, $x$ is the
weight vector of the controller and $N$ is how many times we evaluate the controller.
Thus, the maximum likelihood estimation of the rate 
parameter
of the distribution is given by
\begin{align}
\hat{\lambda}_{exp} = \frac{1}{\hat{\mean}}
\end{align}
A confidence interval is found by the following
\begin{align}
\frac{2N}{\hat{\lambda}_{exp}\chi^{2}_{1-\frac{\alpha}{2},2N}}
<
\frac{1}{\lambda_{exp}}
< 
\frac{2N}{\hat{\lambda}_{exp}\chi^{2}_{\frac{\alpha}{2},2N}}
\end{align}
However, for these experiments, an approximation for a $95\%$
confidence on lower and upper bound 
of the rate parameter $\lambda$ is used
\begin{align}
{\lambda_{exp}}_{low} &= 
\hat{\lambda}_{exp} \left( 1 - \frac{1.96}{\sqrt{N}} \right)\\
{\lambda_{exp}}_{upp} &= 
\hat{\lambda}_{exp} \left( 1 + \frac{1.96}{\sqrt{N}} \right)
\end{align}
By this, the $95\%$ confidence interval for the mean $\mean$ is
\begin{align}
\frac{1}{{\lambda_{exp}}_{low}} < \mean < \frac{1}{{\lambda_{exp}}_{upp}}
\end{align}
When a controllers score is presented as "$s \pm p$" this means 
it has an empirical mean score of $s$ and a real mean that is with 
$95\%$ likelihood within $s \pm p$. \\
Similar confidence intervals are described in the works by Amine Boumaza \citep{boumaza2009}.




\section*{Motivation}
In many challenging applications of machine learning systems, the
learning signals are sparse, unspecific, and/or delayed, for instance
in autonomous robotics or in man-machine interaction, but also when
learning to play games. Supervised
learning cannot be used directly in such a case, but the task can be
cast into a reinforcement learning (RL) problem \citep{littman:15}. Reinforcement
learning is learning from the consequences of interactions with an
environment without being explicitly taught. Because the performance
of standard RL techniques is falling short of expectations, there is a
need to explore new RL algorithms. In this project, we will look at
evolutionary direct policy search for RL.


In particular, we will employ the covariance matrix adaptation
evolution strategy (CMA-ES, \citep{hansen:01,hansen2011}). The CMA-ES has been shown to be highly efficient for episodic RL
(e.g., by \cite{heidrich-meisner:09}).

The cross-entropy method is regarded as state-of-the-art for learning
to play the game of Tetris \citep{szita:06,thiery:09}. The goal of
this project is to perform an unbiased comparison of the cross-entropy
method and CMA-ES for learning Tetris.





\section{Scope and limitations \label{section:scope}}


The experiments will be carried out on the simplified version of
Tetris using the MDPTetris software found at \cite{mdptetris},
which is the same simulator used by Scherrer et al. in \cite{scherrer2009:b} among 
other authors in various papers.
This software already have the well known feature sets
implemented, so we will not ourselves extend any of the features.
The source code of the Tetris simulator is used as-is, and is therefore 
not altered prior to running the experiments. 
For comparing the optimizers, the Shark\footnote{See \url{http://image.diku.dk/shark/} or \cite{shark08}
} library will be used. This library already contains an
implementation of the CMA-ES optimizer, but lack the 
Cross-Entropy. Therefore, a part of this thesis will
be to implement and document the Cross-Entropy method in Shark.






\section{Previous work}

Over the time, numerous researchers has tried different feature 
sets and applied various optimizers to find the best 
possible Tetris controllers. The features used are typically
ones that attempt to mimic the board conditions that would
normally catch the attention of a human player, such as
how high the overall pile of pieces is and how many holes 
the board has. In \citep{scherrer2009:b} table 1, a table 
presents some feature sets used throughout various publications
on the subject. In later works, many authors have had success
with applying evolutionary stochastic search methods for tuning 
the weights of the feature sets towards
efficient controllers. For the purpose of this thesis,
we are in particular addressing the
Cross-Entropy method described in detail in \citep{cetut2014} and the
Covariance Matrix Adaption Evolution Strategy (CMA-ES), described 
in \citep{hansen2011}. The particular Cross-Entropy method applied 
is the one described in \citep{szita:06} as the "Noisy Cross Entropy Method".\\

\begin{figure}[h!]
\begin{center}
\begin{tabular}{| l | p{8cm} |}
\hline
\textbf{Feature} & \textbf{Description}\\
\hline
Landing height & The height of the piece when it lands\\
\hline
Eroded piece cells & Number of rows cleared in the last move
times the number of bricks cleared from the last move\\
\hline
Row transitions & Number of horizontal cell transitions\\
\hline
Column transitions & Number of vertical cell transitions \\
\hline
Holes & Number of empty cells covered by a full cell\\
\hline
Board wells & Cumulative sum of cells to the depth of
the board wells.\\
\hline
\end{tabular}
\end{center}
\caption{features of the Dellacherie controller \label{table:dellfeat}}
\end{figure}


Currently, many researchers have proposed numerous 
feature sets and multiple 
optimization methods have been explored. 
A controller often referred to is the Dellacherie controller 
\citep{fahey}. This controller was hand-tuned by trial and error,
and did originally, on a regular non-simplified Tetris game, achieve an average of
660 000 lines. The same feature set (see figure \ref{table:dellfeat}) is 
often incoorporated in later works when optimizing controllers. An earlier
feature set is the set proposed by \citep{Bertsekas} referred to as Bertsekas and
Tsitsiklis features. In 2006, Szita and L\H{o}rincz \citep{szita:06} applied the Cross-Entropy
method using the Bertsekas and Tsitsiklis features. They report that using no noise,
their controller converged at 300 000 lines on average. 
The best result reported in \citep{szita:06}
is when decreasing noise is applied, 
in which the controller's score exceeded 800 000 lines. 
However, in a later paper, using Dellacherie, 
Bertsekas and two selfdefined features achieved 
35.000.000 lines $\pm 20\%$  \citep{scherrer2009}.\\
\\
Creating a Tetris-controller is a NP-complete problem, 
where we want to find a strategy which maximizes 
the average score. Most researchers utilize three 
general approaches to create policies.

\begin{itemize}
\item Handwritten controllers
\item Reinforcement learning approaches
\item Optimization algorithms
\end{itemize}

But as seen in \citep{scherrer2009}, 
a combination of the methods can yield good results, 
where Thiery and Scherrer employed Dellacherie's 
handwritten policy and used CE to optimize it.


\section*{Approach}
The following strategy will be utilized
\begin{itemize}
\item Configure a system with the necessary packages and installations to solve the project problem.
\item Design and implement an agent for Tetris. 
\item Apply CMA-ES algorithm to train an agent for playing Tetris. (CMA-ES is already implemented in SHARK library)
\item Implement a version of the CE algorithm. (Algorithm is not included in the SHARK library)
\item Apply the implemented CE algorithm to train an agent for playing Tetris.
\item Run a set number of simulations to acquire data for statistical analysis of performance between the two algorithms.
\ Conduct statistical analysis to determine if one of the algorithms performs better at training an agent for playing Tetris than the other.
\end{itemize}

\section*{Requirements}

\subsection*{Running environment}
The implementation will be tested and run 
on a UNIX system, namely Ubuntu 14. However, 
the implementation does not contain 
any platform specific assets ad should run on 
other platforms as well. 

\subsection*{Simulation configurations}
Without initial knowledge about the time consumption of simulations 
needed to evaluate agents, expectations are that the 
experiment will be optimized for running in few configurations
that matches similar experiments conducted by others.

\subsection*{Documentation}
The implementation of the CE algorithm in the SHARK 
library will aspire to be similar to the general 
documentation style of the SHARK library.\\
Alongside, the configuration and the execution of the 
experiment will be documented as part of the report.

\section*{Supervision}
The project group will meet with project the supervisor once every two weeks or on special request, during the project span, to discuss/guide on project difficulties.\\
Furthermore, a weekly progress email will be sent to inform the project supervisor of the developments and general progress of the project.

\section*{Learning Goals}
\subsection*{Reinforcement learning}
Investigate how to apply reinforcement learning to the task of playing Tetris. Documented by the application of CMA-ES and CE algorithms to optimize an agent for playing Tetris, and further by description in the final report.

\subsection*{Neural Network}
Learn how other researchers have approached designing Tetris agents, and
how these are tuned with various algorithms. This will be documented in the 
report as a littereture review and reproduction of results from related works.

\subsection*{CMA-ES and CE}
Learn the fundamentals of genetic evolutionary algorithms
and in particular how CMA-ES end CE works. 
Documented in separate theoretical sections of the final report, 
describing how they function in the project implementation.

\bibliography{BSc2014}[]
\bibliographystyle{plain}


\end{document}

\title{BSc Project Description\\\textbf{Learning to Play Tetris using
    the Covariance Matrix Adaptation
    Evolution Strategy}}
\date{\today}
\maketitle

\section{Time plan}

\subsection{Unofficial - pre-schedule}
This is the time before the official start 
of our bachelor project, i.e. now.\\\\
Setup Shark: Done, on UNIX systems.\\\\
Setup MDPTetris: before 19-10-2015\\\\
Sample CMA-ES simulations,
to get a feel of complexity and runtime: before 16-11-2015

\subsection{Official bachelor schedule}
Upload contract: 20-11-2015\\\\
Uplaod Synpsis: 04-12-2015\\\\
Implement CE for Shark/Tetris: before 10-12-2015\\\\
Setup both experiments: before 15-12-2015\\\\
Upload report: 18-01-2016\\\\
Defense: (28 | 29)-01-2016


\section{Problem statement}
In comparison between the Cross-entropy Method and
Covariance Matrix Adaption Evolution Strategy, which of the
two performs better in training an agent for playing Tetris?

\section{Learning goals}

\subsection{Reinforcement learning}
Investigate how to apply reinforcementlearning 
to the task of playing Tetris.

\subsection{Neural network}
Learn how to setup and apply Neural Networks as an agent for playing
Tetris.

\subsection{CMA-ES and CE}
Learn the fundamentals of genetic evolutionary algorithms
and in particular how CMA-ES end CE works.

\section{Goals}

\subsection{Descriibe and implement Cross-Entropy}
Obtain an implementation of the CEM (Cross-Entropy Method) either 
by implementation or by usage of third-party source. 

\subsection{Implement interface between Shark and Tetris}
Find a suitable tetris playing agent which can be optimized 
by using either CMA-ES or CEM. Implement interface between 
the trainable agent and a Tetris simulator.

\subsection{Compare the two methods}
Run each of the optimization methods and compare
the results to determine weather one outperforms the 
other.

\section{Limitations}

\subsection{Running enviroment}
The implementation will be tested and run 
on a UNIX system, namely Ubuntu 14, however, 
the implementation should be able to run on 
any Unix distribution with the applied setup as in the project. 

\subsection{Simulation configurations}
Without inital perceptions of the time simulations 
needed to evaluate agents, expectations are that the 
experiemnt will be optimized for running a single configuration
set thoroughly.

\subsection{Documentation}
The configuration and the execution of the 
experiment will be documented, however, documntation 
of the code will be of lower priority than the remaining tasks. 



\end{document}

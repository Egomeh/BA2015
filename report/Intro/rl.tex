\subsection{Reinforcement learning \label{RL}}

In the field of Machine Learning, the sub field of reinforcement learning
deals with training agents to behave in an environment through
trial and error. The environment typically consists of a set of states
between which an agent can transition through a set of actions.
The reinforcement learning methods relates somewhat 
to supervised and unsupervised learning models. In unsupervised 
learning, the agents are never given a feedback on their actions
and must attempt to derive some pattern without decisive responses.
In supervised leaning, the agent is trained from labelled data.
In this way the agent will always, for any action, receive 
feedback of whether the committed action was desirable or not.
Reinforcement learning appears somewhat similar to supervised learning.
In reinforcement learning, the agent will commit actions in an environment
and receive rewards based on those. The main difference between supervised
learning and reinforcement learning is that in reinforcement learning,
the agents cannot necessarily tie a correctness 
to a specific action as some action, albeit yielding reward,
may cause relative loss of reward in the future. An example of how this applies 
to artificial agents is the game Tic Tac Toe. If supervised learning
is applied, the playing agent will for each move in the game know
if the move was considered good or bad. In reinforcement learning however,
the only feedback the agent will have is rewards that a given 
to the agent when reaching certain states of the game. 
Often in the case of games, the 
time frame naturally falls into episodes in which the 
goal of the agent is to accumulate as much reward as possible 
before the episode ends. When the agent interacts 
with the environment, it transitions between different states through 
actions. To determine what action to take, the agent will use
a function that the yields a reward given to the agent 
if the action is taken. 
The goal is then to choose a policy, that when followed by the agent,
gives the highest possible cumulative reward over time. 
It's important to notice that the reward 
for an action may be negative, and will in such case actively discourage 
the agent for taking given actions.
The environment, in which the agent appears, often 
or at least for simplicity, has an exit state
in which the agent is terminated, and the cumulative reward 
is revealed. The action transitions may 
not be deterministic and if they are not,
it's required that the probabilities of transitions
remain during the entire episode \citep{Carr}. If the state space 
is large enough, it becomes infeasible to attempt to compute 
the commutative reward when taking actions, as this would require
a thorough search of a large space. Thus, in situations 
with large state spaces, such as Tetris, one must choose 
an approach for finding the best possible policies by other
means than exhaustive search. The methods used in reinforcement learning 
does specify how the agent should change its policy according to 
results from earlier episodes, and the specific methods used in
the context of Tetris is explained in further detail in later sections.














\section{Introduction \label{sec:intro}}

This thesis will cover an experimental approach 
in comparing two state-of-the-art stochastic search
algorithms in the context of learning Tetris. 
The algorithms in consideration
are the Cross-entropy method and 
the Covariance Matrix Adaption Evolution Strategy (CMA-ES
for short). The algorithms are each characterized by their
ability to search in rough multidimensional 
search spaces that offers little possibility 
for analytical approaches.
The algorithms are applied to the task of learning to play Tetris. 
Most often, when machine learning
techniques are applied to problems such as games, conventional
learning methods fall short due to both  very large number of 
possible actions in the games and highly unpredictable mappings 
between actions and their long-term consequences. The individual playthroughs
in Tetris are considered episodic tasks, where some agent is placed
in the game environment and interacts with the game following some
strategy that will hopefully result in a high score. 
The agent will make decisions on how to react 
to the game while playing, and optimizing the agent is hence a task of
finding well performing policies for the agent to follow. Yet, as mentioned,
deciding what makes a policy good can be very difficult problems 
with large state spaces like in games. When applying reinforcement learning methods to learning
the policies, conventional full graph traversal approaches typically do not suffice 
due to infeasible computation times. We will cover how
Tetris can be formulated in way that allows  the Cross-entropy method 
and CMA-ES to search for policies, and we will experimentally 
attempt to obtain an empirical perception of whether of the two algorithms
are preferable in this scenario.

\subsection{Reinforcement learning \label{RL}}

In the field of machine learning, the subfield of reinforcement learning
deals with training agents to behave in an environment through
trial and error. The environment typically consists of a set of states that cover all
scenarios the agent can encounter. The agent then has a set of actions
that it can use to make transitions between the states. 
The reinforcement learning methods relates somewhat 
to supervised models.
In supervised learning, every decision is tied to a predefined value
that defines how desirable the given action is. This way, in supervised learning,
the agent will always at any point have very detailed feedback signals from its actions.
In the context of actions and state transitions, an agent that learns trough
supervised learning will for every action taken know the absolute consequence
immediately after committing it.
Reinforcement learning is in many ways similar to supervised learning. 
The main difference between the two models is that agent does not receive 
an immediate response to its actions. Instead, the agent will receive a reward
when reaching certain states. The objective of the agent is then to
accumulate as much reward as possible. This differs from supervised learning
as the agent cannot know if reaching high amount of reward in the next state
will cause negative consequences in the future. Thus, the only decisive 
feedback given to the agent about its overall performance is 
how much cumulative reward it achieved at the end of a game.
In the game of Tetris, the reward is a direct mapping to 
the score achieved by the agent.
An example of how this two learning models apply to the simple game of
Tic Tac Toe. If supervised learning
is applied, the playing agent will for each move in the game know
if the move was considered desirable or not. 
In reinforcement learning however, only feedback the agent receives
is whether the games ended in a win, draw or a loose. Hence, the agent
cannot rely on signals during the game to adjust it's strategy.\\
\\
Often in the case of games, the 
time frame naturally falls into playthroughs in which the 
goal of the agent is to accumulate as much reward as possible 
before the game ends.
When the agent interacts 
with the environment, it uses a predefined set of actions
that each alters the state of the environment. In the case of 
Tic Tac Toe, the environment is the current state of the game, 
in terms of which markers are occupying spaces on the board.
The actions are the options to place a marker
on an empty space.
When deciding which action to carry out, the agent will use
a policy that maps the state of the environment to an action.
In Tic Tac Toe, a policy might be a table that has an entry for each possible
state of the board that maps to where the agent should place its marker.
The goal of the agent is then to choose a policy that maps states to actions 
in a way that
gives the highest possible cumulative reward at the end of a playthrough. 
Rewards in states can be negative and cause a 'penalty' to the agent if it reaches
such state. However, in the case of Tetris the agent cannot loose score,
and will therefore never be penalized.
Games often have a terminal state at which the game ends, which is 
also the case for Tetris and Tic Tac Toe. It's easy to see how the
Tic Tac Toe cannot last forever as it has 9 spaces, and for each turn in the game,
a space is occupied. However, the reason for why Tetris cannot play forever
lies within the fact that there exists a sequence of pieces that will
with certainty cause the player to loose, and if the player plays for 
long enough, this sequence will eventually occur. Thus, the agent must
attempt to acquire as much reward as possible before reaching the terminal 
state. 
When the agent chooses an action it cannot necessarily know what state
the action leads to, and in games this uncertainty is often present.
In Tic Tac Toe, the action taken by the agent will always with certainty 
lead to a known state, namely the board configuration where the marker is placed.
However, in Tetris, part of the state is the piece that is currently falling.
Thus, when the agent chooses a place to drop the piece, it has full control
over where to drop it, however it cannot decide what the next piece is. 
As there are 7 different piece
types in Tetris, each action can lead to 7 other states, each with a predefined probability.
It should also be noted that these probabilities for the individual transitions
must remain fixed across the entire playthrough of the game \citep{Carr}.
To find the best policy, there are a number of methods to pick from,
and the most obvious is a full traversal of the entire state space.
If one can afford to compute all rewards acquired from all possible 
actions in all possible states, one can choose the policy 
that gives the best cumulative reward. However, if the state space is
large, such computations become infeasible. The state space in Tetris
is indeed far too large to comprehensively walk through.
Due to this, one must use other ways of discovering good policies.
The methods used in reinforcement learning 
specify how the agent should change its policy according to 
results from earlier playthroughs, and the specific methods used in
the context of Tetris is explained in further detail in later sections.




\subsection{Learning Tetris \label{sec:learningTetris}}

On the topic of reinforcement learning, a widely used benchmark
for learning algorithms are designing agents 
for playing the classical video game of Tetris. Tetris is an 
appealing benchmarking problem due to it's complexity. The 
standard games plays on a board made from a grid that is
10 cells wide and 20 cells tall. As the game progress, differently
shaped pieces fall from the top of the board. 
When a row on the board is fully occupied by pieces, the line
is removed, all lines above it moved one line down and a score
point is given to the player. If a cell above the 20 rows first is
occupied, the game ends. The task of the player is to move
and rotate the falling pieces in a way that yields the highest 
score before the game ends.\\
\\
Tetris is indeed a hard task to computationally optimize, as
the game has a very high number of board configurations estimated to be
$10^{59}$ \citep{scherrer2009}. In relation to reinforcement learning,
an agent that plays Tetris is placed in an environment with a set of 
states far too large to exhaustively explore, and a set of transitions
that are stochastic. Because of this
complexity, a common approach 
in the literature is to use 
\textit{one-piece controllers}\footnote{Agents and controllers
both refer to artificial players.}, such as described in 
\cite{scherrer2009:b}. These controllers are only aware of
the current board state and the currently falling piece.
Hence, the policy of the agent is only to greedily choose
the action that transitions to the most rewarding state
from a single piece, which is typically less than 80 states\footnote{
Assuming 4 possible rotations and 20 columns
in which the piece may be dropped.}.
Using these controllers, the search space is reduced 
to only looking at the current board, and the possible 
places to drop the piece. \\
The game used for the benchmarking is a simplified version of Tetris,
in which 
controllers need only to decide in what column to drop the current
piece, and what orientation the piece should have when dropped.
Thus, the simplified version of Tetris differs from the 
original game mainly in two significant aspects. 
First, the controller is 
disallowed to move the piece horizontally while the piece 
is falling. Second, the controller has 'infinite'
time to make its decision on where to drop the current piece.
This way, the game only progress in discrete time steps
when the controller takes
an action, whereas the classical game runs continuously 
regardless of the players actions.
Thus, the controller cannot take advantage of moving the piece 
during the fall, but is not restricted by the time limitations.
This is however a common way of benchmarking Tetris playing agents
\citep{scherrer2009}\\
\\
When the controllers decide which action to take, it will
simulate each of the possible actions and choose the one that
leads to the most favourable board state. To evaluate the board 
state, the controller uses a set of features that defines 
various qualities of the board, and associate a weight to each 
feature. This means that the efficiency of the controller 
is determined by the features the controller is aware of
and how heavily they are weighted. This allows
the controller with $n$ features to be expressed as an 
$n$ dimensional real-valued vector, with one dimension 
per feature, and the value in that dimension the weight.
An often referred to controller is the Dellacherie's controller, 
as described in \cite{scherrer2009}. This controller
takes six features of the board into account, seen in table 
\ref{table:dellfeat} on page \pageref{table:dellfeat}. In relation 
to reinforcement learning, the policy of the agent is the feature set
combined with the weights of each feature. 
When using the \textit{one-piece controllers}, the policy remains 
the same during the episode, and the learning is based on updating
the policy from concluded games, rather than picking up on signals
that occur in the game as it plays.







\subsection{Goals of the thesis}

Both the Cross-entropy method and the CMA-ES have been used 
for learning Tetris with \textit{one piece controllers}, 
%but as mentioned, 
yet to our knowledge, only little effort has been put into 
comparing the two algorithms. This thesis will explore
how the two methods compare against each other under similar
conditions. Therefore, we will use a set of features for the 
Tetris controllers among those
commonly used, and compare how the two optimizing algorithms 
differ. The goal however is not to find a controller that 
outperforms existing controllers, but only to investigate 
how the Cross-entropy method and CMA-ES differs when learning Tetris
with similar configurations.\\
\\
In this thesis, we will explore how the two state-of-the-art
optimization algorithms, Cross-entropy method and CMA-ES, differ when 
applied to the task of playing Tetris.\\
\\
The Shark library \citep{shark08} contains a
working implementation of the CMA-ES 
algorithm. The Cross-entropy method 
is not present in the Shark library and thus, 
a part of the thesis is to implement it ourselves to resemble
the implementation in
the work of other researchers. To document the 
soundness of the implemented Cross-entropy method, 
we will replicate the experiment in \citep{scherrer2009} and 
verify that we obtain the similar results.
Then, we will benchmark CMA-ES and the Cross-entropy method against each other 
to determine if one yields better optimization results than the other.


%\subsection{Scope and limitations \label{section:scope}}


The experiments will be carried out on the simplified version of
Tetris using the MDPTetris software found at \citep{mdptetris},
which is the same simulator used by Scherrer et al. in \citep{scherrer2009:b} 
among other authors in various papers.
This software already have the well known feature sets
implemented, so we will not ourselves extend any of the features.
The source code of the Tetris simulator is used as-is, and is therefore 
not altered prior to running the experiments. 
For comparing the optimizers, the \shark library will be used. 
This library already contains an
implementation of the CMA-ES optimizer, but lacks the 
Cross-Entropy. Therefore, a part of this thesis will
be to implement and document the Cross-Entropy method in Shark.








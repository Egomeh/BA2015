\section{Previous work}

Over the time, various researchers has tried various feature 
sets and applied various optimizers to find the best 
possible Tetris controllers. The features used are typically
ones that attempt to mimic the board conditions that would
normally catch the attention of a human player, such as
how high the overall pile of pieces is and how many holes 
the board has. In \cite{scherrer2009:b} table 1, a table 
presents some feature sets used throughout various publications
on the subject. In later works, many authors have had success
with applying evolutionary stochastic search methods for tuning 
the weights of the feature sets towards
efficient controllers. For the purpose of this thesis,
we are in particular concerned with the 
Cross-Entropy method described in detail in \citep{cetut2014} and the
Covariance MAtrix Adaption Evolution Strategy (CMA-ES), described 
in \cite{hansen2011}. For the Cross-Entropy method,
a certin variant described in \cite{szita:06} using noise
is the one we will apply in the experiments.\\
\\
Currently, many feature sets exists, as well as optimizations for those have been
investigated. A controller often refered to is the Dellacherie controller 
(fahey 2003 \citep{fahey}). This controller was hand tuned by trial and error,
and did originally, on a regular non-simplified Tetris game, acheive an average of
660 000 lines. The same feature set (see figure \ref{table:dellfeat}) is 
often incoorporated in later works when optimizing controllers. An earlyer
featureset is the set proposed by \citep{Bertsekas} refered to as Bertsekas and
Tsitsiklis features. In 2006, Szita and L\H{o}rincz (\citep{szita:06}) applied the Cross-Entropy
method using the Bertsekas and Tsitsiklis features. They report that using no noise,
their controller converged at 300 000 lines on average. The best result reported in \citep{szita:06}
is when decreasing noise is applied, which results in the controller exceeding 800 000 lines.



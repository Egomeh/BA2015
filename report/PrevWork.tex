\section{Previous work}

Over the time, various researchers has tried various feature 
sets and applied various optimizers to find the best 
possible Tetris controllers. The features used are typically
ones that attempt to mimic the board conditions that would
normally catch the attention of a human player, such as
how high the overall pile of pieces is and how many holes 
the board has. In \cite{scherrer2009:b} table 1, a table 
presents some feature sets used throughout various publications
on the subject. In later works, many authors have had success
with applying evolutionary stochastic search methods for tuning 
the weights of the feature sets towards
efficient controllers. For the purpose of this thesis,
we are in particular concerned with the 
Cross-Entropy method described in detail in \citep{cetut2014} and the
Covariance Matrix Adaption Evolution Strategy (CMA-ES), described 
in \cite{hansen2011}. For the Cross-Entropy method,
a certain variant described in \cite{szita:06} using noise
is the one we will apply in the experiments.\\

\begin{figure}[h!]
\begin{center}
\begin{tabular}{| l | p{8cm} |}
\hline
\textbf{Feature} & \textbf{Description}\\
\hline
Landing height & The height of the piece when it lands\\
\hline
Eroded piece cells & Number of rows cleared in the last move
times the number of bricks cleared from the last move\\
\hline
Row transitions & Number of horizontal cell transitions\\
\hline
Column transitions & Number of vertical cell transitions \\
\hline
Holes & Number of empty cells covered by a full cell\\
\hline
Board wells & cumulative sum of cells to the depth of
the board wells.\\
\hline
\end{tabular}
\end{center}
\caption{features of the Dellacherie controller \label{table:dellfeat}}
\end{figure}

Currently, many feature sets exists, as well as optimizations for those have been
investigated. A controller often referred to is the Dellacherie controller 
\citep{fahey}. This controller was hand tuned by trial and error,
and did originally, on a regular non-simplified Tetris game, achieve an average of
660 000 lines. The same feature set (see figure \ref{table:dellfeat}) is 
often incoorporated in later works when optimizing controllers. An earlier
feature set is the set proposed by \citep{Bertsekas} referred to as Bertsekas and
Tsitsiklis features. In 2006, Szita and L\H{o}rincz \citep{szita:06} applied the Cross-Entropy
method using the Bertsekas and Tsitsiklis features. They report that using no noise,
their controller converged at 300 000 lines on average. The best result reported in \citep{szita:06}
is when decreasing noise is applied, in which the controller's score exceeded 800 000 lines. However, in a later paper, using Dellacherie, Bertsekas and two selfdefined features achieved 35.000.000 lines $\pm 20\%$  \citep{scherrer2009}.\\
\\
Creating a Tetris-controller is a NP-complete problem, where we want to find a strategy which maximizes the average score. Most researchers utilize three general approaches to create policies.

\begin{itemize}
\item Handwritten controllers
\item Reinforcement learning approaches
\item Optimization algorithms
\end{itemize}

But as seen in \citep{scherrer2009}, a combination of the methods can yield good results, where Thiery and Scherrer employed Dellacherie's handwritten policy and used CE to optimize it.

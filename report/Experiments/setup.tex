\subsection{Setup}

When executing the experiments, various parameters each have 
an impact on the final result of the learning curve. Thus, the parameters
are adjusted, first to match the experiments run by other researchers, 
and later to conduct as fair as possible comparisons between 
the Cross-entropy method and CMA-ES.\\
\\
% agents
The amount of vectors sampled in each generation $\populationSize$
has a high impact on the algorithm performance. By setting $\populationSize$
high, more policies are evaluated per iteration, and leads to a more thorough 
exploration of the search space. Thus the higher $\populationSize$ increases the
chances of finding a better mean for the next iteration.
However, higher $\populationSize$ also results in the
need for more evaluations per iteration. The goal for 
tuning this parameter is then
to set $\populationSize$ high enough to ensure 
exploration of good solutions, and yet 
low enough to avoid unnecessary evaluations.\\
In the implementation of CMA-ES from \shark , 
the algorithm  itself determines
the value of $\populationSize$ according to the 
size of the search space. 
The Cross-entropy method however, does not seem to have a 
general rule for this parameter,
so this value is manually adjusted to fit the 
problem as well as possible.\\
\\
% offspring
As both of the optimizing algorithm uses a subset of the sampled vectors
from a generation to update the distribution parameters, the number of 
offspring $\offspringNumber$ influences how the next generation is sampled.
By setting the value too high, the algorithm risks ceasing to progress any 
further since the updated mean would be too close to the previous one to 
significantly make a difference. By setting the value too low,
the risk of reaching a local optimum increases since the high-scoring agents
might have reached their high performance by chance.\\
The CMA-ES itself manages setting $\offspringNumber$ and the Cross-entropy method
is set according to the problem. Most authors that uses the Cross-entropy method for Tetris
sets the offspring size to $10\%$ of population size, that is 
$\offspringNumber = \lfloor 0.1 \cdot \populationSize \rfloor $.\\
\\
% Number of games per iteration
The number of games, $\numberOfEvaluations$, 
is the number of games  which each agent 
plays in each iteration. An agent's score is defined as the 
mean of the score of these 
$\numberOfEvaluations$ games.
We want this value low as possible, because as with the number of
agents, $\populationSize$, The number of games, $\numberOfEvaluations$, 
is another major factor in the run-time of the algorithm.
As Tetris is stochastic by nature, the score deviates a lot, 
even when the
same agent with the same policy plays multiple games. 
Hence, when assessing the true
performance of a policy it's rarely enough to play just few games. Thus, setting 
$\numberOfEvaluations$ high increases the likelihood of correctly choosing the best 
agents, yet, it also causes longer run times of the experiments.\\
\\
% Noise factor
Specific to the Cross-entropy method, 
most authors report that the performance of the 
algorithm increases dramatically when the sampling 
distribution is associated with
a noise term. The different types of 
noise are described in section \ref{CrossEntropy}.
The noise term is adjusted in order to 
prevent the algorithm from reaching a local optimum.
The current research shows that noise terms of $\noise^{(\generation)} = 4$ and 
$\noise^{(\generation)} = max \left( 5 - \generation / 10 \right)$ 
\citep{thiery:09} produces the best results.
The constant noise (such as $\noise^{(\generation)} = 4$) ensures that the algorithm
never settles in a too small area from which it samples, and forces it to explore
solutions that are further away from the mean. The further the 
algorithm progresses, 
the less noise is assumed needed, as the mean should approach a global optimum. to
address this, the linear decreasing noise 
is applied as it will lower the noise term
as the algorithm progresses.\\
\\
For the various experiments, these 
parameters will be tuned for the specific purpose 
at hand. In the verification of the Cross-entropy method, the parameters are set 
to match those reported in similar papers (\cite{thiery:09}, \cite{szita:06}).
In the comparison of the two algorithms, the parameters will be set such that 
the Cross-entropy method operates under as 
similar conditions as CMA-ES, to ensure an unbiased 
comparison.






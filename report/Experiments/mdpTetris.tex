\subsection{MDP-Tetris}

When running the experiments, the source code of the MDP-Tetris
\citep{mdptetris} was used to emulate the Tetris games.
The source code is accompanied with files that describe the
various existing features. These files contains the identifiers of 
each feature to use, as well as two numbers respectively describing 
the agents reward function and how to evaluate a game over state. 
The number for the reward function has remained unchanged at $0$ 
during all experiments. The "game-over" evaluation was for the
Bertsekas feature set initially set to $0$. Setting the 
"game-over" evaluation to $0$ means that the agent will not 
distinguish between regular moves and moves that results in losing
the game. When running the experiments with this setting, a large portion
of the agents never exceeded a zero mean score. However, setting the value
to $-1$, meaning that a "game-over" move yields $-\infty$ reward, 
none of the experiments got stuck on only zero scores. An example
of the layout of the feature file can be seen in figure \ref{fig:featfile}.
\begin{figure}[h!]
\centering
\begin{lstlisting}
0    <- Describes the reward function
-1   <- Actions leading to game over is avoided at all cost
22   <- The policy contains 22 features
8 0  <- The feature with id 8 initially has weight 0
...  <- The remaining 21 features
\end{lstlisting}
\caption{Example of a file that describes a feature set. \label{fig:featfile}}
\end{figure}

\subsubsection{Featureset}
\comment{Write stuff about featuresets and the difference they can make}

\subsubsection{Game complexity}
\comment{Write stuff aboutGame complexity and the difference it can make}




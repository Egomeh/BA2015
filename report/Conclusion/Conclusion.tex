\section{Conclusion}

To reach our goal of gathering empirical evidence if either the Cross-entropy method
and the CMA-ES differ in performance when applied to Tetris, we implemented the Cross-entropy method in Shark and tested it against the pre-implemented version of CMA-ES
already in the Shark library. Our implementation of the Cross-entropy 
method has been accepted by the Shark development team, and is now part of the code base.
The widely used MDPTetris software was used to emulate the
Tetris games, and was used as the optimization problem for the optimization algorithms to solve.
The Cross-entropy method searches by using a Gaussian 
distribution, and therefore is stochastic in nature. Due to this, 
the verification of the algorithm was carried out as a comparison with other 
researchers works of the same algorithm applied to the same problem, namely Tetris.\\
\\
After confirming that the Cross-entropy method does produce similar results as 
what we've seen in our reference papers, we conducted an initial comparison between the
Cross-entropy method and CMA-ES. The initial experiments used the default settings for CMA-ES
from Shark,
and the settings for the Cross-entropy method were the ones claimed to be optimal by other 
researchers. This initial experiment, revealed quite the opposite of our expectations 
according to our hypothesis, which was that the CMA-ES would outperform the Cross-entropy method
but at slower convergence. This experiment lead to the need of finding out whether the algorithms
could perform better with other settings. To find the better settings, a more difficult version
of Tetris was used to reduce the learning time of the algorithms. To thoroughly explore 
the impact of the different adjustable parameters more than 6,000 individual runs of the 
algorithms contributed to gain insight in how to best configure the two algorithms. These 
optimal settings were applied in a final comparison experiment.\\
\\
The results from the final experiment showed that a result that aligns better with our
hypothesis. The convergence of CMA-ES is heavily prolonged by the optimal settings
but does eventually reach scores that climbs slightly beyond those produced by the
Cross-entropy method. However, the final score of the two algorithms lies so close that
we cannot firmly differentiate between the two, and hence, cannot conclude that either of the
algorithms are better than the other. However, the results from our experiments 
suggests that the Cross-entropy method reaches its 
highest scores much faster than CMA-ES under optimal settings, thus in conclusion
when learning Tetris using the MDPTetris platforms' approach the Cross-entropy method 
stands out as the preferable choice for learning Tetris.


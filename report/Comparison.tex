\section{Comparison between CMA-ES and CE}

\subsection{Global comparison settings}
In all papers used for reference we haven't seen any experiments with different population
and offspring sizes presented side-by-side. However, in our upcoming experimental settings
CMA and CE are most likely to have different population and offspring sizes, which means
that we can't compare them based on iterations/generations. Instead, using the  
\textit{number of agents} as comparison reference, equal terms are secured for both algorithms 
in regards to learning potential. As one of the adjustments to the algorithms are tuning the 
population size, it's required that the frame of reference is invariant.\\
Regarding the graphs themselves, the comparison graphs' x-axis shows the total number 
of Tetris games evaluated, $\sum_{i = 1}^{\generation} \populationSize$. Meanwhile
the y-axis still represents the mean agent's score of the iteration $\generation$.

\subsection{Initial comparison - Bertsekas}
For the initial comparison we use the Bertsekas featureset, since the same featureset
was used for verifying the Cross Entropy implementation. Furthermore, others researchers
has used the Bertsekas featureset as a benchmarking standpoint \citep{thiery:09} \&
\citep{szita:06}.\\
The goal of this comparison is to get an initial idea of how the Shark implementation of
CMA compares to Cross Entropy.\\

\textbf{Results}

\comment{- WHAT results did the comparison yield}\\
\comment{- Show results from comparison experiments}


\textbf{Analysis and discussion}

\comment{- WHY did we get these results}\\
\comment{- Discuss the results of the comparison experiments}


\subsection{Dellacherie comparison}

\comment{Indledenede tekst}

\textbf{Results}

\comment{- WHAT results did the comparison yield}\\
\comment{- Show results from comparison experiments}


\textbf{Analysis and discussion}

\comment{- WHY did we get these results}\\
\comment{- Discuss the results of the comparison experiments}
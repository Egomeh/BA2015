\section{Comparison between CMA-ES and CE}

\subsection{Global comparison settings}
In all papers used for reference we haven't seen any experiments with different population
and offspring sizes presented side-by-side. However, in our upcoming experimental settings
CMA and CE are most likely to have different population and offspring sizes, which means
that we can't compare them based on iterations/generations. Instead, using the  
\textit{number of agents} as comparison reference, equal terms are secured for both algorithms 
in regards to learning potential. As one of the adjustments to the algorithms are tuning the 
population size, it's required that the frame of reference is invariant.\\
Regarding the graphs themselves, the comparison graphs' x-axis shows the total number 
of Tetris games evaluated, $\sum_{i = 1}^{\generation} \populationSize$. Meanwhile
the y-axis still represents the mean agent's score of the iteration $\generation$.

\subsection{Initial comparison - Bertsekas}
For the initial comparison we use the Bertsekas featureset, since the same featureset
was used for verifying the Cross Entropy implementation. Furthermore, others researchers
has used the Bertsekas featureset as a benchmarking standpoint \citep{thiery:09} \&
\citep{szita:06}.\\
The goal of this comparison is to get an initial idea of how the Shark implementation of
CMA compares to Cross Entropy.\\

\textbf{Results}

\comment{- More statistical results, multilevel model, etc.}\\

Using Cross Entropy with the constant noise setting and CMA with an initial step-size
of $0.5$, we get the following results, seen in figure \ref{fig:CMA_VS_CE_00}.\\

\begin{figure}[H]
\begin{tikzpicture}
\cmaCePlot
\end{tikzpicture}
\caption{Initial comparison between CMA-ES and Cross Entropy \label{fig:CMA_VS_CE_00}}
\end{figure}

As it can be observed in figure \ref{fig:CMA_VS_CE_00} CMA converges faster,
but seems to reach a local optimum around 2,000 agents evaluated. Meanwhile CE has a 
slower convergence but reaches a better local optimum compared to CMA, around 5,500
agents evaluated. In detail CMA on average reaches a score 50,000 rows, and
CE reaches a score of 100,000.\\

\textbf{Analysis and discussion}

These results clearly defy our initial hypothesis, namely we estimated that
CMA would clearly outperform CE, due to CMA's more sophisticated method of 
searching. One reason for this outcome could possible be that
CMA has a very little population size compared to Cross Entropy,
which could be a decisive lack as the evaluation function is non-deterministic with 
a very high variance. These results could indicate that we need to tweak the CMA
configuration, as the CMA local optimum is reached quickly compared to Cross Entropy.
More specifically we might need to adjust the number games played gamer per agent,
since the low population size and increasing variance makes the CMA converge on a
local optimum. But as it can be seen in figure \ref{fig:CMA_VS_CE_00}, the
fast convergence suggests that one game per agent is sufficient until the
score begins to stall at around 1,000-2,000 agents evaluated. It would seem that 
as the CMA-ES reaches a score of around 50,000, variance of the objective function 
appears to render the algorithm incapable of effectively distinguish performance 
of agents. Therefore, to increase the perception of the agents’ performs, 
more evaluations per agent could be necessary at higher scores.\\
\\
As no experiment so far has yielded consistent scores of more than around 200,000 
on average, one explanation could be that the feature set itself poses a limit for 
the optimization algorithms. To address this issue, further experiments will include 
other feature sets. Also, the specific task is not to find a good controller for Tetris, 
but to compare how well each algorithm performs. To reduce runtimes, the Tetris games 
played can be configured to be harder to play. This includes possible reduction of 
board size and posing higher chances of occurrence of pieces that are difficult to place.\\
\\
Therefore, based on the current results, conducting more experiments to discover optimal
settings for CMA-ES seems appropriate.

\subsection{Configuration of CMA}
\comment{- Write about adjustments made to CMA because of the previous experiment}

\subsection{Dellacherie comparison}

\comment{Indledenede tekst}

\textbf{Results}

\comment{- WHAT results did the comparison yield}\\
\comment{- Show results from comparison experiments}


\textbf{Analysis and discussion}

\comment{- WHY did we get these results}\\
\comment{- Discuss the results of the comparison experiments}

\subsection{Dynamic games played per agent}

\comment{Indledenede tekst}

\textbf{Results}

\comment{- WHAT results did the comparison yield}\\
\comment{- Show results from comparison experiments}


\textbf{Analysis and discussion}

\comment{- WHY did we get these results}\\
\comment{- Discuss the results of the comparison experiments}

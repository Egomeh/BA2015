\section{Experiments}

This section will explore our experimental setup, our verification of CE and 
and a analysis/discussion of the results.

\subsection{Setup}

When executing the experiments, various parameters each have 
impact on the final result of the learning curve. Thus, the parameters
are adjusted, first to match the experiments run by other authors, and later
to conduct as fair as possible comparisons between Cross-Entropy and 
CMA-ES.\\
\\
% agents
The amount of vectors sampled in each generation $\populationSize$
has obvious impact on the algorithm performance. By setting $\populationSize$
high, more policies are evaluated per iteration, and leads to a more thorough 
exploration of the search space. Thus the higher $\populationSize$ increases the
chances of finding a better mean for the next iteration.
However, higher $\populationSize$ also results in the
need for more evaluations per iteration. The goal for tuning this parameter is then
to set $\populationSize$ high enough to ensure exploration of good solutions, and yet 
low enough to avoid unnecessary evaluations.\\
In the implementation of CMA-ES used in \shark , the algorithm  itself determines
the value of $\populationSize$ according to the size of the search space. 
Cross-Entropy however, does not seem to have a general rule for this parameter,
so this value is manually adjusted to fit the problem as well as possible.\\
\\
% offspring
As both of the optimizing algorithm uses a subset of the sampled vectors
from a generation to update the distribution parameters, the number of 
offspring $\offspringNumber$ influences how the next generation is sampled.
By setting the value too high, the algorithm risks ceasing to progress any 
further since the new mean would be too close to the previous mean to 
significantly move the mean. By setting the value too low,
we risk reaching a local optimum since the high-scoring agents 
might have been lucky in "one-time" occurring situations.\\
Once again, the CMA-ES itself manages setting $\offspringNumber$ and Cross-Entropy
is set according to the problem. Most authors that uses Cross-Entropy for Tetris
sets the offspring size to $10\%$ of population size, that is 
$\offspringNumber = \lfloor 0.1 \cdot \populationSize \rfloor $.\\
\\
% Number of games per iteration
The number of games, $\numberOfEvaluations$, is the number of games  which each agent 
has to play in each iteration. An agent's score from a given
iteration/generation is defined as the mean of the score of these 
$\numberOfEvaluations$ games.
We want this value low as possible, because as with the number of
agents, $\populationSize$, The number of games, $\numberOfEvaluations$, 
is another major factor which 
impacts the run-time of the algorithm.
As Tetris is stochastic by nature, the score deviates a lot, even when the
same agent with the same policy plays multiple games. Hence, when assessing the true
performance of a policy it's rarely enough to play just few games. Thus, setting 
$\numberOfEvaluations$ high increases the likelihood of correctly choosing the best 
agents, yet, it also causes longer run times of the experiments.\\
\\
% Noise factor
Specific to the Cross-Entropy method, most authors report that the performance of the 
algorithm increases dramatically when the sampling distribution is associated with
a random noise. The different types of noise are described in section \ref{CrossEntropy}.
The noise term is adjusted in order to prevent the algorithm from reaching a local optimum.
The current research shows that noise terms of $\noise_\generation = 4$ and 
$\noise_\generation = max \left( 5 - \generation / 10 \right)$ \citep{thiery:09}.
The constant noise (such as $\noise_\generation = 4$) ensures that the algorithm
never settles in a too small area from which it samples, and forces it to explore
solutions that are further away from the mean. The further the algorithm gets, 
the less noise is assumed needed, as the mean should approach a global optimum. to
address this, the linear decreasing noise is applied as it will lower the noise term
as the algorithm progresses.\\
\\
For the various experiments, these parameters will be tuned for the specific purpose 
at hand. In the verification of the Cross-Entropy, the parameters are set 
to match those reported in similar papers (\cite{thiery:09}, \cite{szita:06}).
In the comparison of the two algorithms, the parameters will be set such that 
the Cross-Entropy operates under as similar conditions as CMA-ES, to ensure an unbiased 
comparison.


\subsection{Verification of CE}
Because the \shark library already contains an implementation of 
CMA-ES, but not an implementation of CE, we extended the library 
with our own implementation of the algorithm. 
This ensures as many similar conditions as possible for 
the two optimization algorithms as possible.\\
In order to verify the correctness of the implementation, 
we used the same experiments as used by 
Christophe Thiery and Bruno Scherrer \citep{thiery:09}. 
These experiments were used be Thiery and Scherrer to 
verify their own CE implementation with various types of noise correction. 
Therefore, we will utilize the same experiments to verify our 
own contribution to the \shark library, by trying to achieve the same results.\\
\\
The setup is mirrored as in the paper \citep{thiery:09}, 
with 100 agents ($\populationSize = 100$) per iteration. 
The 10 best agents ($\offspringNumber = 10$) will be used 
to generate the new gaussian distribution. After each iteration, 
an agent with the mean weights from the $\offspringNumber$ best agents, 
play 30 games.\\
The paper does not specify the number of games 
played with each agent in an iteration to evaluate their fitness. 
Thus it is assumed they also play 30 games.\\
\\
\comment{GRAPHS - graps?}\\
\\
\comment{ANALYSIS/COMPARE - does graphs compare good?}\\
\\
\comment{Conclusion - is CE good enough? cyka blyat}

\subsection{Results}

What were the results of the experiment? This section will
present theoutcome of the experiment, or refer to data.

\subsection{Analysis}

How do CE and CMA-ES compare from our data? Can we say anything 
which performs better?


\subsection{Implementation of the Cross-entropy method}

As we wish to compare the Cross-entropy method 
with the CMA-ES algorithm, we needed to have both algorithms
in the same environment. We choose to use the Shark 
library\footnote{\url{http://image.diku.dk/shark/}}
as the framework for our experiments. Shark is a machine 
learning library written in C++ and is shipped with various tools
for formulating machine learning problems and a flexible 
interface for applying algorithms for these. Another appealing
feature of Shark is the already implemented CMA-ES algorithm.
The Shark library does however not include an implementation of
the Cross-entropy method. As part of this thesis, we have contributed
an implementation of the Cross-entropy method to the Shark library.
In Shark, there is a collection of algorithms grouped under the label
'Direct Search' algorithms. These algorithms are designed to solve 
problems like the one described in \ref{ProblemFormulation}. In this
set of algorithm, the CMA-ES implementation exists. Our Cross-entropy method 
is written to match the style of the CMA-ES code. As the algorithm
works much in similar ways, the Cross-entropy method vastly resembles the CMA-ES, 
but with the core components of replaced.\\
\\
Our Cross-entropy method implementation has been accepted by the Shark development team and is 
by now part of the official library, which can be found at 
\url{https://github.com/Egomeh/Shark}, whereas the header of the 
implementation can be found at 
\url{https://github.com/Shark-ML/Shark/blob/master/include/shark/Algorithms/DirectSearch/CrossEntropyMethod.h}.\\
\\
The algorithm itself is modelled after the ones we've encountered 
in the literature, in particular, a description of the algorithm is 
present in \citep{thiery:09}. Thiery and Scherrer are the authers of the 
MDPTetris platform used in our experiments, and the paper presents 
results from their implementation of the Cross-entropy method. As the 
software is shipped with a working implementation of the Cross-entropy method, written in C,
this code has strongly influenced in our implementation as guidelines for
the details of the implementation. The results from the paper is then further used
to verify that our implementation of the Cross-entropy method is reasonable and working in 
the like the ones we claim to resemble. The specifics of the validation process
are described in detail in section \ref{varifyofce}. The paper only presents the
graphs from their experiments, and single examples from the final iterations
of their experiments. We tried to contact Bruno Scherrer in the hope that 
the data might still be available, however, he wrote back that the data from the
original experiment is no longer stored. As a result of this, in our verification,
observing similar graphs and a similar span in results has to suffice as 
empirical evidence that our implementation work as desired. 
The algorithm also need to be able to work as a general-purpose 
search algorithm, such that it can be applied to problems of the same kind as 
the CMA-ES in Shark. To make sure of this, the code that merged with the 
Shark library includes a handful of unit tests. The tests chosen were some
of that CMA-ES is tested with. In these tests, the algorithms are applied to some 
function, and if the found solution is better than a certain threshold, the 
algorithm passes the test. The Cross-entropy method managed to pass a portion 
of those also solved by CMA-ES, albeit some with a more tolerant threshold.




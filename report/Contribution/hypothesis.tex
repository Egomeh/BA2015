\subsection{Hypothesis}

Prior to any execution of experiments it's assumed that, purely based 
on a theoretical standpoint, that the CMA-ES algorithm will be able to 
find better solutions than the Cross-entropy method. The assumption is based 
on the fact that CMA-ES is, to a higher degree, able to adjust to the 
search space. Yet, as the Cross-entropy method is a simpler algorithm that needs 
to adjust fewer parameters. Hence, another expectation is that the Cross-entropy
method will converge faster than CMA-ES, but at lower scores.\\
\\
The CMA-ES does both have a higher number of adjustable parameters, but does also
maintain information on prior iterations unlike the Cross-entropy method.
Due to holding the information about prior iterations, it's perhaps a reasonable
expectation that the algorithm will take longer to adjust properly to the
problem. However, the CMA-ES does by default only have a little population size 
compared to the Cross-entropy method. The default population size of the CMA-ES is
typically either a forth or half of number of dimensions of the problem's search space.
In our case with the MDPTetris, the number of features in a feature set never exceeds
22, thus limiting the default population size very little compared to the Cross-entropy method's 
100. In the CMA-ES tutorial \citep{hansen2011}, the little population is said to
work well for deterministic problems. Yet, as the scores in Tetris is heavily
influenced by noise that is likely to cause faulty ranking of candidate solution,
the small population size could prove to be a critical shortcoming.
It is however expected that even after adjusting both algorithms to perform as 
well as possible, the CMA-ES will prove to converge slower than the Cross-entropy method, but
with better final results.\\
\\
As a final remark on the hypothesis, we must recognize that the experiments can 
end up rendering little difference between the two algorithms. The reason for 
this concern lies within the Tetris application itself. As the platform used for benchmarking 
Tetris controllers mainly appears to focus on the choice of the feature set as the primary
focus of optimization. Often when improving the controllers, the common approach is to
construct the feature sets such that the controller is aware of as many critical 
details of the game as possible. Our focus lies on the optimization algorithms,
and other research does indeed show that adjustments to these algorithms can 
improve the solutions. In spite of this, we cannot prior to the experiments know how much
the choice of algorithm impacts the results, and we may face the situation that 
the algorithms only play a marginal role compared to the feature set.

\section{Goals of the thesis}

Both the Cross-Entropy and the CMA-ES methods has been used 
in learning Tetris with \textit{one piece controllers}, but as 
mentioned, to our knowledge, only little effort has been put into 
comparing the two methods. This thesis will explore
how the two methods compare against each other under similar
conditions. Therefore, we will use a set of features among those
commonly used, and compare how the two optimizing algorithms 
differ. The goal however is not to find a controller that 
outperforms existing controllers, but only to investigate 
how the Cross-Entropy and CMA-ES differs when learning Tetris
with similar controllers.\\
\\
In this paper, we will explore how the two state-of-the-art
optimization algorithms, Cross-Entropy and CMA-ES, differ when 
applied to the task of playing Tetris.\\
\\
The Shark library \citep{shark08} contains a
working implementation of the CMA-ES 
algorithm. However, the Cross-Entropy method 
is not present in the Shark library and thus, 
a part of the thesis is to implement it ourselves according to 
the other researchers work. To document the 
soundness of the implemented CE method, 
we will replicate the experiment in \citep{thiery:09} and 
verify that we obtain the similar results.\\
Then, we will benchmark CMA-ES and CE against each other 
to determine if one yields better optimization results than the other.\\
\\
The experiments will be carried out on the simplified version of
Tetris using the MDPTetris software found at \citep{mdptetris},
which is the same simulator used by Scherrer et al. in \citep{scherrer2009:b} 
among other authors in various papers.
This software already have the well known feature sets
implemented, so we will not ourselves extend any of the features.
The source code of the Tetris simulator is used as-is, and is therefore 
not altered prior to running the experiments. 
For comparing the optimizers, the \shark -library will be used. 
This library already contains an
implementation of the CMA-ES optimizer, but lacks the 
Cross-entropy method which we present an implementation of.
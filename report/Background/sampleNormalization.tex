\subsection{Normalization of samples \label{normalSamples}}
As mentioned by some authors \citep{boumaza2009}, the vector that
describes the agent can very well be normalized such that the vector
is a point that lies on the $\dimensions$-dimensional hypersphere.\\
\\
The reason for this lies in the nature of the evaluation function.
When the controller chooses an action, it will evaluate all the 
possible actions possible with the current piece. It will use the 
value function $\valueFunction$ of each state $\gameState_i$ and 
choose the state with the highest value from the value function.
Thus, if the states are ordered such that:
\begin{align}
\valueFunction \left(  \gameState_1 \right) 
> \dots 
> \valueFunction \left( \gameState_\populationSize \right)
\end{align}

The agent then chooses the action that transitions from the current state 
to state $\gameState_1$.\\
Since the value function assess the state by the following:
\begin{align}
\valueFunction (\gameState ) &= 
\sum_{i=1}^{\dimensions} \weight _{i}\feature _{i}(\gameState )
\end{align}

Then scaling the input of the agent, the weight vector $\weight$, by a
number $a \in \mathbb{R}, \ a > 0$ the assessment is changed by:
\begin{align}
\sum_{i=1}^{\dimensions} a \weight _{i}\feature _{i}(\gameState ) &= 
a\sum_{i=1}^{\dimensions} \weight _{i}\feature _{i}(\gameState )\\
&= a \valueFunction \left( \gameState \right)
\end{align}

And the ordering remains:
\begin{align}
a \valueFunction \left(  \gameState_1 \right) 
> \dots 
> a \valueFunction \left( \gameState_\populationSize \right)
\end{align}

Thus the order of the value functions of 
each state does not change, and the same $\gameState_1$
is still chosen for any $a \in \mathbb{R}, \ a > 0$.\\
\\
To verify this, the Tetris objective function was executed with the
same vector and the same seed for the random generator with a scale
$a \in \{0.1, 0.2,0.3, \dots, 9.8,9.9,10.0\}$, and the agent scored 
exactly the same for each scale.\\
\\
This can be used in experiments for various reasons. As reported 
in \citep{boumaza2009}, normalizing the samples will 
prevent CMA-ES from diverging in step size,
and it can prevent loosing precision if the magnitude of weight 
vector becomes larger than feasible for the used floating 
point number and avoids size limitations. However, brief experiments 
with normalization of samples did not appear to have any effect 
on either CMA-ES nor the Cross Entropy algorithm. Thus, albeit 
being aware of the normalization option, this was not implemented
in the actual experiments.





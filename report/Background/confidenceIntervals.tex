\subsection{Assesment of controller performance}
The performance of a one-piece controller has a very high variation,
and is in other research verified to be exponentially distributed.
As a result of the high variance of the controllers, the performance 
of single controllers are often presented along with a confidence interval
for the estimated mean score of the controller. The estimated mean of 
the controllers score is calculated by
\begin{align*}
\hat{\mean} &= \sum_{i=1}^{\populationSize} \individual_i
\end{align*}
Thus, the maximum likelihood estimation of the rate 
parameter\footnote{Note that $\lambda$ is in this context not,
the population size but instead the rate parameter for the
exponential distribution.}
of the distribution is given by
\begin{align*}
\hat{\lambda} = \frac{1}{\hat{\mean}}
\end{align*}
A confidence interval is found by the following
\begin{align*}
\frac{2\populationSize}{\hat{\lambda}\chi^{2}_{1-\frac{\alpha}{2},2\populationSize}}
<
\frac{1}{\lambda}
< 
\frac{2\populationSize}{\hat{\lambda}\chi^{2}_{\frac{\alpha}{2},2\populationSize}}
\end{align*}
However, for these experiments, an approximation for a $95\%$
confidence on lower and upper bound 
of the rate parameter $\lambda$ is used
\begin{align*}
\lambda_{low} &= 
\hat{\lambda} \left( 1 - \frac{1.96}{\sqrt{\populationSize}} \right)\\
\lambda_{upp} &= 
\hat{\lambda} \left( 1 + \frac{1.96}{\sqrt{\populationSize}} \right)
\end{align*}
By this, the $95\%$ confidence interval for the mean $\mean$ is
\begin{align*}
\frac{1}{\lambda_{low}} < \mean < \frac{1}{\lambda_{upp}}
\end{align*}
When a controllers score is presented as "$s \pm p$" this means 
has an empirical mean score of $s$ and a real mean that is with 
$95\%$ likelihood within $s \pm p$. \\
\comment{Add reference to exponential distribtutions}

\subsection{Assesment of controller performance \label{sec:confidenceIntervals}}
The performance of an one-piece controller has a very high variation,
and is in other research verified to be exponentially distributed. Hence, the mean
performance of a controller $m$ is exponentially distributed with an unknown 
rate parameter  $\lambda_{exp}$\footnote{Note that $\lambda_{exp}$ refers to the rate parameter
of an exponential distribution, and not the population size of the optimization algorithms.}.
As a result of the high variance of the controllers, the performance 
of single controllers are often presented along with a confidence interval
for the estimated mean score of the controller. The estimated mean of 
the controllers score is computed as
\begin{align}
\hat{\mean} &= \sum_{i=1}^{N} \fitnessFunction \left( \individual \right)
\end{align}
Where $\hat{m}$ is the oberved mean score of the controller, $x$ is the
weight vector of the controller and $N$ is how many times we evaluate the controller.
Thus, the maximum likelihood estimation of the rate 
parameter
of the distribution is given by
\begin{align}
\hat{\lambda}_{exp} = \frac{1}{\hat{\mean}}
\end{align}
A confidence interval is found by the following
\begin{align}
\frac{2N}{\hat{\lambda}_{exp}\chi^{2}_{1-\frac{\alpha}{2},2N}}
<
\frac{1}{\lambda_{exp}}
< 
\frac{2N}{\hat{\lambda}_{exp}\chi^{2}_{\frac{\alpha}{2},2N}}
\end{align}
However, for these experiments, an approximation for a $95\%$
confidence on lower and upper bound 
of the rate parameter $\lambda$ is used
\begin{align}
{\lambda_{exp}}_{low} &= 
\hat{\lambda}_{exp} \left( 1 - \frac{1.96}{\sqrt{N}} \right)\\
{\lambda_{exp}}_{upp} &= 
\hat{\lambda}_{exp} \left( 1 + \frac{1.96}{\sqrt{N}} \right)
\end{align}
By this, the $95\%$ confidence interval for the mean $\mean$ is
\begin{align}
\frac{1}{{\lambda_{exp}}_{low}} < \mean < \frac{1}{{\lambda_{exp}}_{upp}}
\end{align}
When a controller's score is presented as "$s \pm p$" this means 
has an empirical mean score of $s$ and a real mean that is with 
$95\%$ likelihood within $s \pm p$. \\
Similar confidence intervals are described in the works by Amine Boumaza \citep{boumaza2009}.

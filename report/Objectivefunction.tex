\section{Objective function}

When the Tetris simulator plays Tetris, the internal decision process
of the tetris controller is configured with a set of parameters that remain
fixed across the entire game. These parameters are the weights associated 
with each feature function that is considered by the controller. The features $\feature _{i}$
each map a state of the game $\gameState$ to a real value. An overview of the exact mappings
can be seen in table 1 in \citep{scherrer2009:b}. How much each of these mappings
should count in on the final evaluation of the board is determined by $\weight _{i}$
denoting the weight of the $i$-th feature. Finally, the function to assess the value 
of current board state $\valueFunction$, with $\dimensions$ features functions present, can be expressed as:

\begin{align*}
\valueFunction (\gameState ) &= \sum_{i=1}^{\dimensions} \weight _{i}\feature _{i}(\gameState )
\end{align*}

For each possible action the controller may take, it simulates 
the action and evaluate the value of the board for the new state. 
The final action is the one that yields the state of the highest value.
The performance of the controller is hence directly tied to the 
features and weights in the evaluation function. To adjust these controllers,
one can either change the set of used feature functions, or as the 
optimization algorithms will do, change the weighting of the features.
For the experiments, the feature sets remain fixed, and the task of the
optimization algorithm applied is to tune the set of weights in order 
to maximize the performance of the controller.\\
\\
The objective function then accepts a vector of values for each weight
and returns a single real number that represents how well the controller
performed while influenced by the input weighting.




% evaluating a specific agent
Now that the new generation has been created, it's time for each agent to play g games of Tetris.\\
For a given Tetris game, we say that a game consists of a range of states. A state is defined as, when an agent has to place a given piece, and it has to choose where to place the piece on the board. Placing the piece is called an action, and since there are multiple ways to place the piece on the board, we have a range of possible of actions. In a given state the agent has to try each possible action, and the action with the highest evaluation, is the action the agent executes in the current state. 
When evaluating each possible action, the objective function S is used. As shown in the input-section, the objective function is defined as:

\begin{align*}
\valueFunction (\gameState ) &= \sum_{i=1}^{\dimensions} \weight _{i}\feature _{i}(\gameState )
\end{align*}

Where $F_i$ is the i'th feature function and w\_i is the i'th weight associated with the corresponding i'th feature function. More specifically w\_i is the value of the i'th dimension of the agent's vector.
A specific action then gets evaluated by computing S(x). The main factor of the evaluation value is how the action changed the board layout , since each feature function F\_i (x) looks at different "types" of board layouts (insert ref. to feature policy table).

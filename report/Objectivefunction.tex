\section{Objective function}

When the Tetris simulator plays Tetris, the internal decision process
of the tetris controller is configured with a set of parameters that remain
fixed across the entire game. These parameters are the weights associated 
with each feature function that is considered by the controller. The features $\feature _{i}$
each map a state of the game $\gameState$ to a real value. An overview of the exact mappings
can be seen in table 1 in \citep{scherrer2009:b}. How much each of these mappings
should count in on the final evaluation of the board is determined by $\weight _{i}$
denoting the weight of the $i$-th feature. Finally, the function to assess the value 
of current board state $\valueFunction$, with $\dimensions$ features functions present, can be expressed as:

\begin{align*}
\valueFunction (\gameState ) &= \sum_{i=1}^{\dimensions} \weight _{i}\feature _{i}(\gameState )
\end{align*}

For each possible action the controller may take, it simulates 
the action and evaluate the value of the board for the new state. 
The final action is the one that yields the state of the highest value.
The performance of the controller is hence directly tied to the 
features and weights in the evaluation function. To adjust these controllers,
one can either change the set of used feature functions, or as the 
optimization algorithms will do, change the weighting of the features.
For the experiments, the feature sets remain fixed, and the task of the
optimization algorithm applied is to tune the set of weights in order 
to maximize the performance of the controller.\\
\\
The objective function then accepts a vector of values for each weight
and returns a single real number that represents how well the controller
performed while influenced by the input weighting.


